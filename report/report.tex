\documentclass{article}

\usepackage[a4paper,margin=2.5cm]{geometry}
\usepackage{graphicx}
\usepackage{amsmath}
\usepackage{amssymb}
\usepackage{amsthm}
\usepackage{amsfonts}
\usepackage{parskip}


\title{Object Oriented Programming Report}
\author{Giorgio Grigolo}
\date{Semester 1, 2022}

\begin{document}

\maketitle
\begin{abstract}
This report is a summary of the Object Oriented Programming course.
\end{abstract}


\newpage

\section{Minesweeper --- A C++ Implementation}

\subsection{Introduction}
Minesweeper is a logic puzzle video game genre generally played on personal
computers. The game features a grid of clickable squares, with hidden ``mines''
scattered throughout the board. The objective is to clear the board without
detonating any mines, with help from clues about the number of neighboring
mines in each field.

In this report, we will implement Minesweeper with the help of
ncurses\footnote{Ncurses is a programming library providing an application with a 
terminal-independent screen-painting and keyboard-handling facility in a
text-mode environment.} and C++.

\subsection{The Game}
The game is played on a board of tiles, each of which is either a mine or
empty. The player is initially presented with a board of all empty tiles, and
must use logic to deduce the locations of the mines. The player can click on a
tile to reveal it. If the tile is a mine, the player loses. If the tile is
empty, the tile will be revealed, and if it has no neighboring mines, all of
its neighboring tiles will be revealed as well. If the tile has neighboring
mines, the number of neighboring mines will be displayed on the tile. If the player
marks all of the mines, the player wins.


\subsection{The Implementation}

\subsubsection{UML}

% \begin{figure}[h]
% \centering
% \includegraphics[width=0.8\textwidth]{mine_UML.png}
% \caption{UML Diagram}
% \end{figure}


\subsubsection{The Board}

The board is a 2D array of tiles. The board is initialized with a given number
of mines, and the rest of the tiles are empty. The board is displayed using
ncurses. The board is updated when the player clicks on a tile.


\subsection{The Design}

\subsection{The Code}

\subsection{The Results}



\end{document}
